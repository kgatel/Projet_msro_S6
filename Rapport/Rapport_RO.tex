\documentclass[12,french]{report}
\usepackage{geometry}
\geometry{vmargin=3cm, hmargin=3cm}
\usepackage[T1]{fontenc}
\usepackage[utf8]{inputenc}
\usepackage[french]{babel}
\usepackage{graphicx}
\usepackage{amsmath}
\usepackage{amssymb}
\usepackage{sectsty}
\usepackage{authblk}
\usepackage{algpseudocode}
\usepackage{algorithm}
\usepackage{xspace}
\usepackage{mathtools}
\usepackage{mathrsfs}
\usepackage{enumitem}
\usepackage{titlesec}
\usepackage{hyperref}
\usepackage{xcolor}
\usepackage[justification=centering]{caption}
\usepackage{float}
\usepackage{tabto}

\usepackage{listings}
\usepackage{cleveref}

\renewcommand{\lstlistingname}{Code}
%\renewcommand{\figurename}{Fig.}

\lstdefinestyle{chstyle}{%
backgroundcolor=\color{gray!12},
basicstyle=\ttfamily\small,
showstringspaces=false,
numbers=left}

%\AddThinSpaceBeforeFootnotes
%\FrenchFootnotes

\titleformat{\chapter}[hang]{\bf\Huge}{\thechapter.}{2pc}{}
\titlespacing*{\chapter}{10pt}{0pt}{40pt}[0pt]
\newcommand{\HRule}{\rule{\linewidth}{0.5mm}}

\providecommand{\keywords}[1]{\textbf{\textit{Keywords:}} #1}
\bibliographystyle{apalike}

\usepackage{hyperref}

\begin{document}
\hypersetup{pdfborder=0 0 0}

\begin{titlepage}

\begin{center}
	\vspace*{\stretch{1}}
	\textsc{{\LARGE Institut national des sciences appliquées de Rouen} \\ 			\vspace{6mm} {\Large INSA de Rouen}} \\
	\vspace{5mm}
	\includegraphics[width=0.4\textwidth]{./Images/insa}\\[1.0 cm]

	\textsc{\Large Projet MSRO GM3 - Vague 2 - Sujet 3}\\[0.6cm]

	% Title
	\HRule \\[0.1cm]
	{ \Huge \bfseries Traitement du signal \\ Etude de deux filtres linéaires}\\[0.2cm]
	\HRule \\[0.95cm]

	\includegraphics[width=0.8\textwidth]{./Images/Page_de_garde}\\[0.9 cm]

	% Author and supervisor
	\begin{minipage}{0.4\textwidth}
		\begin{flushleft} \large
			\emph{Auteurs:}\\
			Thibaut \textsc{André-Gallis} \\
			{\small\href{mailto:thibaut.andregallis@insa-rouen.fr}{thibaut.andregallis@insa-rouen.fr}} \\
			Kévin \textsc{Gatel} \\
			{\small\href{mailto:kevin.gatel@insa-rouen.fr}{kevin.gatel@insa-				rouen.fr}}
		\end{flushleft}
	\end{minipage}
	\begin{minipage}{0.4\textwidth}
		\begin{flushright} \large
			\emph{Enseignant:} \\
			Natalie \textsc{Fortier} \\
			{\small\href{mailto:natalie.fortier@insa-rouen.fr}								{natalie.fortier@insa-rouen.fr}}
		\end{flushright}
	\end{minipage}
	\vspace*{\stretch{1}}

	\vfill
	{\large 11 Avril 2021}
\end{center}
\end{titlepage}

\tableofcontents

%\listoffigures

\renewcommand{\chaptername}{}
\chapter*{Introduction du problème}

Soit deux filtres $h_k$ et $g_k$ tels que :\\

$$ H(z) = \frac{0.3-0.2z^{-1}+0.4z^{-2}}{1+0.9z^{-1}+0.8z^{-2}} $$
$$ G(z) = \frac{0.2-0.5z^{-1}+0.3z^{-2}}{1+0.7z^{-1}+0.85z^{-2}} $$ \\

Dans un premier temps, on cherche à savoir si ces deux filtres sont réalisables physiquement.\\

On s’intéressera ensuite à la mise en cascade de ces deux filtres pour déterminer s'il agit par équivalence d'un filtre unique.\\

Enfin nous étudierons leur mise en parallèle avec les mêmes objectifs que pour leur mise en cascade.


\chapter{Filtres réalisables ?}

\section{Pôles}

\subsection{$H(z)$}

\vspace{0.25cm}

On a $$ H(z) = \frac{0.3-0.2z^{-1}+0.4z^{-2}}{1+0.9z^{-1}+0.8z^{-2}} = \frac{0.3z^2-0.2z+0.4}{z^2+0.9z+0.8} $$

Dénominateur : $$ D_h= z^2+0.9z+0.8 $$

D'où : $$ \begin{array}{ccl}
\Delta & = & 0.9^2-4*1*0.8 \\
	   & = & -2.39 \\
\end{array} $$

On obtient :
$$\left.\begin{aligned}
	&z_{1Dh} = \frac{-0.9+i\sqrt{2,39}}{2} \\
	&\quad = -0.45 + i\frac{\sqrt{2,39}}{2} \\
\end{aligned}\quad\right|
\quad\left.\begin{aligned}
	&z_{2Dh} = \frac{-0.9-i\sqrt{2,39}}{2}\\
	&\quad = -0.45 - i\frac{\sqrt{2,39}}{2} \\
\end{aligned}\right.$$

Ainsi :
$$ |z_{1Dh}|=|z_{2Dh}|=\sqrt{(-0.45)^2+\left(\frac{\sqrt{2,39}}{2}\right)^2} \simeq 0.89 < 1 $$


$H(z)$ a tous ses pôles à l'intérieur du cercle unité \textbf{donc le filtre est réalisable physiquement}.\\

\subsection{$G(z)$}

\vspace{0.25cm}

On a $$ G(z) = \frac{0.2-0.5z^{-1}+0.3z^{-2}}{1+0.7z^{-1}+0.85z^{-2}} = \frac{0.2z^2-0.5z+0.3}{z^2+0.7z+0.85} $$

Dénominateur : $$ D_g= z^2+0.7z+0.85 $$

D'où : $$ \begin{array}{ccl}
\Delta & = & 0.7^2-4*1*0.85 \\
	   & = & -2.91 \\
\end{array} $$

On obtient :
$$\left.\begin{aligned}
	&z_{1Dg} = \frac{-0.7+i\sqrt{2,91}}{2} \\
	&\quad = -0.35 + i\frac{\sqrt{2,91}}{2} \\
\end{aligned}\quad\right|
\quad\left.\begin{aligned}
	&z_{2Dg} = \frac{-0.7-i\sqrt{2,91}}{2}\\
	&\quad =-0.35 - i\frac{\sqrt{2,91}}{2} \\
\end{aligned}\right.$$

Ainsi :
$$ |z_{1Dg}|=|z_{2Dg}|=\sqrt{(-0.35)^2+\left(\frac{\sqrt{2,91}}{2}\right)^2} \simeq 0.92 < 1 $$


$G(z)$ a tous ses pôles à l'intérieur du cercle unité \textbf{donc le filtre est réalisable physiquement}.

\section{Zéros}

\subsection{$H(z)$}

\vspace{0.25cm}

On a $$ H(z) = \frac{0.3z^2-0.2z+0.4}{(z-z_{1Dh})(z-z_{2Dh})} $$

Numérateur : $$ N_h= 0.3z^2-0.2z+0.4 $$

D'où : $$ \begin{array}{ccl}
\Delta & = & (-0.2)^2-4*0.3*0.4 \\
	   & = & -0,44 \\
\end{array} $$

On obtient :
$$\left.\begin{aligned}
	&z_{1Nh} = \frac{0.2+i\sqrt{0.44}}{2*0.3} \\
	&\quad\quad = \frac{1}{3} + \frac{5}{3}i\sqrt{0.44} \\
\end{aligned}\quad\right|
\quad\left.\begin{aligned}
	&z_{2Nh} = \frac{0.2-i\sqrt{0.44}}{2*0.3}\\
	&\quad\quad = \frac{1}{3} - \frac{5}{3}i\sqrt{0.44} \\
\end{aligned}\right.$$

\subsection{$G(z)$}

\vspace{0.25cm}

On a $$ G(z) = \frac{0.2z^2-0.5z+0.3}{(z-z_{1Dg})(z-z_{2Dg})} $$

Numérateur : $$ N_g= 0.2z^2-0.5z+0.3 $$

D'où : $$ \begin{array}{ccl}
\Delta & = & (-0.5)^2-4*0.2*0.3 \\
	   & = & 0.01 \\
\end{array} $$

On obtient :
$$\left.\begin{aligned}
	&z_{1Ng} = \frac{0.5+\sqrt{0.01}}{2*0.2} \\
	&\quad\quad = \frac{3}{2} \\
\end{aligned}\quad\right|
\quad\left.\begin{aligned}
	&z_{2Ng} = \frac{0.5-\sqrt{0.01}}{2*0.2}\\
	&\quad\quad = 1 \\
\end{aligned}\right.$$\\

\section{Représentation dans le plan complexe (zplane)}

\vspace{0.25cm}
On a :
$$ H(z) = \frac{(z-z_{1Nh})(z-z_{2Nh})}{(z-z_{1Dh})(z-z_{2Dh})} $$

et
$$ G(z) = \frac{(z-z_{1Ng})(z-z_{2Ng})}{(z-z_{1Dg})(z-z_{2Dg})} $$

\begin{figure}[H]
    \begin{minipage}[c]{.50\linewidth}
        \centering
        \includegraphics[width=1\textwidth]{./Images/zplane_H}\\
        \caption{Visualisation des pôles et des zéros du filtre $H$}
    \end{minipage}
    \hfill%
    \begin{minipage}[c]{.50\linewidth}
        \centering
        \includegraphics[width=1\textwidth]{./Images/zplane_G}\\
        \caption{Visualisation des pôles et des zéros du filtre $G$}
    \end{minipage}
\end{figure}\vspace{0.3cm}

\section{Réponse en fréquence}

\begin{figure}[H]
    \begin{minipage}[c]{.50\linewidth}
        \centering
        \includegraphics[width=1\textwidth]{./Images/freqz_H}\\
        \caption{Visualisation de $|H(f)|$ sur [0,0.5]}
    \end{minipage}
    \hfill%
    \begin{minipage}[c]{.50\linewidth}
        \centering
        \includegraphics[width=1\textwidth]{./Images/freqz_G}\\
        \caption{Visualisation de $|G(f)|$ sur [0,0.5]}
    \end{minipage}
\end{figure}\vspace{0.3cm}


\chapter{Mise en cascade}


\chapter{Mise en parallèle}


\chapter*{Conclusion}
\addcontentsline{toc}{chapter}{Conclusion}



\chapter*{Annexe}
\addcontentsline{toc}{chapter}{Annexe}



\end{document}
